\documentclass{article}
\usepackage[UTF8]{ctex}  % 使用中文支持包
\usepackage[a4paper, margin=1in]{geometry}  % 设置纸张大小和边距
\usepackage{anyfontsize}  % 解决字体大小报错问题
\usepackage{fancyhdr}  % 设置页眉、页脚、页码
\usepackage{longtable}  % 支持长表格

\usepackage{amsmath}  % 数学公式支持
\usepackage{cases}  % 支持联立编号

\usepackage{graphicx}  % 插入图片支持
\usepackage{float}  % 设置图片浮动位置
\usepackage{subfigure}  % 插入多图时用子图显示

\usepackage{listings}  % 代码块支持
\usepackage{xcolor}  % 设置代码块颜色

\usepackage[hyphens]{url}  % 支持链接换行
\usepackage{hyperref}  % 超链接支持
\usepackage{lastpage}  % 添加lastpage包
\usepackage{gbt7714}  %国标参考文献
\bibliographystyle{gbt7714-numerical}
\hypersetup{
    hidelinks,
    colorlinks=true,
    allcolors=black,
    pdfstartview=Fit,
    breaklinks=true
}

\title{射线源导论-第九周作业}
\author{\LaTeX\ by\ Jerry\ }
\date{\today}
\pagenumbering{arabic}

\begin{document}
\pagestyle{fancy}

\fancyhead[L]{Jerry}
\fancyhead[C]{射线源导论-第九周作业}
\fancyhead[R]{\today}
\fancyfoot[C]{Page \thepage/\pageref{LastPage}}

\section*{第九周课程作业}

\emph{北京高能同步辐射光源(HEPS)是我国正在建设的第四代同步辐射光源,其电子束能量为6GeV,能量排在Spring-8(8GeV)、APS(7GeV)之后,与ESRF(6GeV)并列第三,束流流强$\approx$200mA,预计于2026年建成并交付用户使用。根据不同应用需求,装置将建设90余条光束线和用户实验站供用户使用,其中一期将建设14条光束线供光给不同用户使用,X射线参数通过调整弯铁参数或者波荡器插入件参数进行调整。}

\emph{A) HEPS用于高能X射线辐射的弯铁长度约0.857米,磁场梯度约为1T,偏转半径约为20米,试计算电子束经过该偏转磁铁辐射总功率、辐射射线特征波长}

\emph{B)ID19束线采用波荡器插入件产生辐射,波荡器周期长度$\lambda_u$ = 22.6mm,$K$ = 2.3, 共174个周期,计算经过该波荡器后辐射的X射线波长(基频光),并估算其带宽;}

A.

% 电子束的辐射功率为$$P = \frac{e^2c}{6\pi\varepsilon_0}\left(\frac{E_e}{mc^2}\right)^4\frac{\beta^4}{R^2} = 3.44 \times 10^7 W$$

单个粒子同步辐射总功率$$P_0=\frac{q^2}{6\pi \varepsilon_0 c^3}\alpha^2\gamma^4=\frac{e^2c}{6\pi \varepsilon \rho^2}\beta^4\gamma^4 = 2.19 \times 10^{-6}W$$

辐射射线特征波长 $$\lambda_c=\frac{4 \pi p}{3 \gamma^3}=5.17\times 10^{-11}m$$

B.

$\theta = 0$时,波长$$\lambda_1 = \frac{\lambda u}{2\gamma^2}\left(1+\frac{k^2}{2}\right)=2.9876\times 10^{-10}m$$

带宽$$\Delta \lambda = \lambda\cdot\frac{1}{N}= = 1.717\times 10^{-12}m$$

\end{document}
