\documentclass{article}
\usepackage[UTF8]{ctex}  % 使用中文支持包
\usepackage[a4paper, margin=1in]{geometry}  % 设置纸张大小和边距
\usepackage{anyfontsize}  % 解决字体大小报错问题
\usepackage{fancyhdr}  % 设置页眉、页脚、页码
\usepackage{longtable}  % 支持长表格

\usepackage{amsmath}  % 数学公式支持
\usepackage{cases}  % 支持联立编号

\usepackage{graphicx}  % 插入图片支持
\usepackage{float}  % 设置图片浮动位置
\usepackage{subfigure}  % 插入多图时用子图显示

\usepackage{listings}  % 代码块支持
\usepackage{xcolor}  % 设置代码块颜色

\usepackage[hyphens]{url}  % 支持链接换行
\usepackage{hyperref}  % 超链接支持
\usepackage{lastpage}  % 添加lastpage包
\usepackage{gbt7714}  %国标参考文献
\bibliographystyle{gbt7714-numerical}
\hypersetup{
    hidelinks,
    colorlinks=true,
    allcolors=black,
    pdfstartview=Fit,
    breaklinks=true
}

\title{射线源导论-第四讲作业}
\author{\LaTeX\ by\ Jerry\ }
\date{\today}
\pagenumbering{arabic}

\begin{document}
\pagestyle{fancy}

\fancyhead[L]{Jerry}
\fancyhead[C]{射线源导论-第四周作业}
\fancyhead[R]{\today}
\fancyfoot[C]{Page \thepage/\pageref{LastPage}}

\section*{第四周课程作业}

\subsection*{一、对于100 MeV电子,当加速梯度达到多大时,电子能量不在增加(辐射能量等于加速能量)?}

加速度方向与电子运动方向一致,则辐射功率 $$P = \frac{q^2}{6\pi\varepsilon c^3}a^2\gamma^6$$

其中,$P$为辐射功率,$q$为电子电荷,$\varepsilon$为真空介电常数,$c$为真空光速,$a$为加速度,$\gamma$为洛伦兹因子。

加速功率 $$P_0 = qEv $$

电子速度与能量关系 $$E_e = \gamma m c^2 \rightarrow \gamma = \frac{E_e}{mc^2} = \frac{E_e}{E_0} = 195.6947$$

加速度$$ a = \frac{qE}{m} $$

考虑到电子能量较高,可以近似认为$$ v \approx c$$

辐射功率等于加速功率,则 $$\frac{q^2}{6\pi\varepsilon c^3}a^2\gamma^6 = qEv \rightarrow \frac{q^2}{6\pi\varepsilon c^3}\left(\frac{qE}{m}\right)^2\gamma^6 = qEc $$

解得$$E = \frac{6\pi\varepsilon m^2 c^4}{q^3 \gamma^6}$$

代入数值,得到$$ E = 4.84 \times {10}^{6} \text{V/m}$$

\subsection*{二、对于1 GeV电子和质子,以半径30米作圆周运动时,每圈辐射损失多少能量?}

加速度方向与电子运动方向垂直,则辐射功率 $$P = \frac{q^2}{6\pi\varepsilon c^3}a^2\gamma^4$$

其中,$P$为辐射功率,$q$为电子电荷,$\varepsilon$为真空介电常数,$c$为真空光速,$a$为加速度,$\gamma$为洛伦兹因子。

对电子,同理有$$\gamma_e = \frac{E_e}{E_{0, e}} = 1956.947$$

近似有速度 $$ v_e \approx c $$

则运动时间 $$ t_e = \frac{S}{v_e} = \frac{2\pi R}{c} = 6.28 \times {10}^{-7} \text{s}$$

加速度$$ a_e = \frac{v_e^2}{R} = \frac{c^2}{R} $$

损失能量 $$ E_{loss, e} = Pt_e = \frac{q^2}{6\pi\varepsilon c^3}a_e^2\gamma_e^4 t_e $$

代入数值,得到 $$ E_{loss, e} = 4.723 \times {10}^{-16} \text{J} = 2.952 \text{MeV} $$

对于质子,同理有$$\gamma_p = \frac{E_p}{E_0} = 1.066 $$

有速度$$ v_p = c \sqrt{1-\frac{1}{\gamma_p^2}} = 0.3466c = 1.04 \times {10}^{8} \text{m/s} $$

则运动时间 $$t_p = \frac{S}{v_p} = \frac{2\pi R}{v_p} = 1.81 \times {10}^{-6} \text{s}$$

加速度$$ a_p = \frac{v_p^2}{R} = 0.120156 \frac{c^2}{R} $$

损失能量 $$ E_{loss, P} = Pt_p = \frac{q^2}{6\pi\varepsilon c^3}a_p^2\gamma_p^4 t_p $$

代入数值,得到 $$ E_{loss, p} = 1.733 \times {10}^{-30} \text{J} = 1.083 \times 10^{-11} \text{eV} $$

\end{document}
