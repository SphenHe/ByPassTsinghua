\documentclass[utf8]{article}
\usepackage[UTF8]{ctex}

\usepackage{amsmath}        %数学公式
\usepackage{amssymb}
\usepackage{cases}          %联立编号
\usepackage{cite}           %引用

\usepackage{graphicx}       %插入图片
\usepackage{float}          %设置图片浮动位置
\usepackage{subfigure}      %插入多图时用子图显示

\usepackage{anyfontsize}    %解决一个奇怪的字体大小报错问题
\usepackage{fancyhdr}       %页眉、页脚、页码
\usepackage[a4paper, margin=1in]{geometry}    %纸张大小
\usepackage{longtable}

\title{\bf\huge 概率论与数理统计 - 作业 1}
\author{Jerry}
\date{\today}

\begin{document}
\maketitle

\section*{T2. }

\subsubsection*{Bertrand 悖论:在一个圆内任意选一条弦,这条弦的弦长长于这个圆的内接等边三角形的边长的概率是多少?}

在Bertrand 悖论中,有三种不同的解法:

a. 在直径上取点然后做垂线产生弦,此时概率为$\frac{1}{2}$

b. 固定弦的一点然后在圆周上取另一点,此时概率为$\frac{1}{3}$

c. 在圆内取点然后根据中点做弦,此时概率为$\frac{1}{4}$

~

在这三种选取方式中,由于所认为的“等可能性”不同,故样本空间不同;同时,将“随机选取”理解为在样本空间内等可能取点,从而导致了悖论的产生。可以看出选择不同的坐标会导致不同的概率分配,因此在定义概率时要事先明确指出样本空间是什么。

\section*{T3. }

甲应该得到150元,乙应该得到50元。

甲乙两人实力相当,我们可以认为获胜的概率均为0.5。在比赛中,第一轮的结果为已知的,并不能对后续的比赛结果产生影响。在等概率的情况下考虑后续的比赛,有三种可能的结果:

a. 甲胜,则比赛结束(概率为1/2)

b. 乙胜,则比赛继续(概率为1/2)

\ \ \ \ b.1. 甲再胜,则比赛结束(概率为1/2*1/2=1/4)

\ \ \ \ b.2. 乙再胜,则比赛结束(概率为1/2*1/2=1/4)

可以得到甲胜的概率为3/4,乙胜的概率为1/4,故甲得200*3/4=150元,乙得50元

\section*{T4. }

假设$A$赢的概率为$p$,获得的利润为$W$,

有$W$的数学期望为:$W=5p-20(1-p)=25p-20\geq0$,解之得$p\geq0.8$,故甲的胜率应大于等于0.8,此人对 A 队获胜的主观概率是80\%。

\section*{T5. }

假设$A$、$B$赢的概率分别为$p$、$q$,获得的利润为$W$

总奖池可以理解为1000元,则可以认为$A$、$B$、$C$的赔率分别为2、10/3、5。

假设有三个投资者分别投资$A$、$B$、$C$,投资的资金为$X$

对$A$有:\[W_A=2pX+0(1-p)(1-X)-X=2pX-X\geq0\Rightarrow p\geq0.5\]

对$B$有:\[W_B=10/3*qX+0(1-q)(1-X)-X=10/3*qX-X\geq0\Rightarrow q\geq0.3\]

对$C$有:\[W_C=5(1-p-q)X+0(p+q)(1-X)-X=5(1-p-q)X-X\geq0\Rightarrow 1-p-q\geq0.2\]

三名投资者均不希望亏损,则当且仅当$p=0.5,q=0.3$时,三个不等式同时成立

故$A$、$B$、$C$赢的概率分别为50\%,30\%,20\%

\section*{T6. }

预测结果:

HTHTTHHTTHHTHTHTTHHHTTTHTHHHTTHHTTTTHHTTHHTTHHTHTH

实际结果:

\begin{longtable}{|l|l|l|l|}
    %\centering
    \caption[]{实际结果统计表格} \\

    % 下面是表头
    \hline 次数(n) & H或T & 频数 n(H) & 相对频数(n(H)/n) \\ \hline
    \endfirsthead

    % 下面数字4的意思是表格的列数
    \multicolumn{4}{l}{{\bfseries \tablename\ \thetable{} -- continued from previous page}} \\
    \hline 次数(n) & H或T & 频数 n(H) & 相对频数(n(H)/n) \\ \hline
    % 注意这里把表头复制了一遍,因为在新的页面也会展示一下表头,不然表格不方便阅读
    \endhead

    \hline \multicolumn{4}{l}{{Continued on next page}} \\ \hline
    \endfoot

    \hline
    \endlastfoot

    % 下面就是真正的表格数据了,注意不用再写表头了
        1 & H & 1 & 1.000000 \\
        2 & H & 2 & 1.000000 \\
        3 & T & 2 & 0.666667 \\
        4 & T & 2 & 0.500000 \\
        5 & T & 2 & 0.400000 \\
        6 & T & 2 & 0.333333 \\
        7 & T & 2 & 0.285714 \\
        8 & T & 2 & 0.250000 \\
        9 & H & 3 & 0.333333 \\
        10 & H & 4 & 0.400000 \\
        11 & H & 5 & 0.454545 \\
        12 & H & 6 & 0.500000 \\
        13 & H & 7 & 0.538462 \\
        14 & H & 8 & 0.571429 \\
        15 & T & 8 & 0.533333 \\
        16 & H & 9 & 0.562500 \\
        17 & T & 9 & 0.529412 \\
        18 & H & 10 & 0.555556 \\
        19 & T & 10 & 0.526316 \\
        20 & T & 10 & 0.500000 \\
        21 & T & 10 & 0.476190 \\
        22 & T & 10 & 0.454545 \\
        23 & H & 11 & 0.478261 \\
        24 & T & 11 & 0.458333 \\
        25 & H & 12 & 0.480000 \\
        26 & H & 13 & 0.500000 \\
        27 & H & 14 & 0.518519 \\
        28 & T & 14 & 0.500000 \\
        29 & T & 14 & 0.482759 \\
        30 & H & 15 & 0.500000 \\
        31 & H & 16 & 0.516129 \\
        32 & H & 17 & 0.531250 \\
        33 & T & 17 & 0.515152 \\
        34 & H & 18 & 0.529412 \\
        35 & T & 18 & 0.514286 \\
        36 & T & 18 & 0.500000 \\
        37 & H & 19 & 0.513514 \\
        38 & T & 19 & 0.500000 \\
        39 & H & 20 & 0.512821 \\
        40 & H & 21 & 0.525000 \\
        41 & H & 22 & 0.536585 \\
        42 & T & 22 & 0.523810 \\
        43 & T & 22 & 0.511628 \\
        44 & T & 22 & 0.500000 \\
        45 & H & 23 & 0.511111 \\
        46 & T & 23 & 0.500000 \\
        47 & H & 24 & 0.510638 \\
        48 & H & 25 & 0.520833 \\
        49 & T & 25 & 0.510204 \\
        50 & T & 25 & 0.500000 \\
\end{longtable}

可以看出相对频数呈现相对稳定的趋势。

对预测的数据做相同的处理,可以发现预测的数据的相对频数在n>10后,始终在0.45与0.55之间波动,且相对频数随着n的增加逐渐趋于0.5,故预测结果与实际结果不存在实质性差异。

\section*{T7. }

我们将病人情况记为(是否投保,身体状况)

(a). A包含以下2种情况:(1,f),(0,f)

(b). B包含以下3种情况:(0,g),(0,f),(0,s)

(c). $B^C+A$包含以下4种情况:(1,g),(1,f),(1,s),(0,f)

\section*{T8. }

$A\cap B+A\cap B^C=A\cap (B+B^C)=A\cap C=A$

\section*{T9. }

(a.1). 证明相等:

$A+B=A\cup B=A\cup ((B-A)+A)=A\cup (B-A)+A\cup =A\cup (B-A)+A$

$\because A\subseteq A\cup (B-A)$

$\therefore A+B=A\cup (B-A)+A=A\cup (B-A)=A+(B-A)$

(a.2). 证明互斥:

$A\cap (B-A)=A\cap (B\cap A^C)=B\cap A\cap A^C=B\cap (A\cap A^C)=B\cap \varnothing=\varnothing$

(b.1). 证明相等:

$A+B=A+(B-A)=AB+AB^C+(B-A)=A\cap B+(B-A)+AB=(A-B)+(B-A)+AB$

(b.2). 证明互斥:

$(A-B)\cap AB=AB^C\cap AB=A\cap (B^C\cap B)=A\cap \varnothing=\varnothing$

$(A-B)\cap (B-A)\subseteq A\cap (B-A)=\varnothing$

$AB\cap (B-A)\subseteq A\cap (B-A)=\varnothing$

\section*{T10. }

%$(A+B)-(A-B)=(A\cup B)\cap (A\cap B^C)^C$

$(A+B)-(A-B)=((A-B)+(B-A)+AB)-(A-B)=((A-B)\cap (A-B))\cup ((B-A)\cap (A-B))\cup (AB\cap (A-B))=\varnothing \cup \varnothing \cup \varnothing= \varnothing$

\section*{T11. }

记$A_1+A_2+A_3+...+A_n=A$

首先将$A_1$分离出,即$A=A_1+A$\textbackslash$A_1=A_1+(A_2+A_3+...+A_n)$\textbackslash$A_1=A_1+A_2$\textbackslash$A_1+A_3$\textbackslash$A_1+...+A_n$\textbackslash$A_1$

同理,再将$A_2$分离出,即$A$\textbackslash$A_1$=$A_2$\textbackslash$A_1+(A$\textbackslash$A_1)$\textbackslash$A_2=A_2$\textbackslash$A_1+(A_3+...+A_n)$\textbackslash$A_2=A_2$\textbackslash$A_1+(A_3$\textbackslash$A_2)$\textbackslash$A_1+...+(A_n$\textbackslash$A_2)$\textbackslash$A_1$

同理,再将$A_3$分离出,……

以此重复下去,则可$A$拆为多个互斥事件的和:

$A=A_1+A_2$\textbackslash$A_1+(A_3$\textbackslash$A_2)$\textbackslash$A_1+((A_4$\textbackslash$A_3)$\textbackslash$A_2)$\textbackslash$A_1+...+(...(A_n$\textbackslash$A_{n-1})$\textbackslash$A_{n-2}...)$\textbackslash$A_1$

\end{document}