\documentclass{article}
\usepackage[UTF8]{ctex}  % 使用中文支持包
\usepackage[a4paper, margin=1in]{geometry}  % 设置纸张大小和边距
\usepackage{anyfontsize}  % 解决字体大小报错问题
\usepackage{fancyhdr}  % 设置页眉、页脚、页码
\usepackage{longtable}  % 支持长表格

\usepackage{amsmath}  % 数学公式支持
\usepackage{cases}  % 支持联立编号
\usepackage{cite}  % 引用支持

\usepackage{graphicx}  % 插入图片支持
\usepackage{float}  % 设置图片浮动位置
\usepackage{subfigure}  % 插入多图时用子图显示

\usepackage{listings}  % 代码块支持
\usepackage{xcolor}  % 设置代码块颜色

\usepackage[hyphens]{url}  % 支持链接换行
\usepackage{hyperref}  % 超链接支持
\usepackage{lastpage}  % 添加lastpage包

\hypersetup{
    hidelinks,
    colorlinks=true,
    allcolors=black,
    pdfstartview=Fit,
    breaklinks=true
}

\setlength{\headheight}{16pt}
\pagestyle{fancy}
\fancyhf{}

\title{\bf\huge 用声音思考的艺术:从4分33秒到青年之歌}
\author{Jerry}
\date{\today}
\pagenumbering{arabic}

\begin{document}

\fancyhead[L]{Jerry}
\fancyhead[C]{用声音思考的艺术:从4分33秒到青年之歌}
\fancyhead[R]{西方弦乐艺术探究}
\fancyfoot[C]{\thepage}

\maketitle

\emph{音乐是用声音进行思维的艺术——朱尔斯·康巴瑞沃(1859-1916)}

\section{前言}

正如朱尔斯·康巴瑞沃所言,“音乐是用声音思考的艺术”这句话概括了音乐的精髓。在我看来,音乐是一种通过听觉而非视觉或者文字来传达思想和情感的媒介。本文希望通过探讨约翰·凯奇的\emph{4'33''}和卡尔海因兹·施托克豪森的\emph{Gesang der Jünglinge}(《青年之歌》)两首曲子,来更深刻的说明这一感悟。

\section{\emph{4'33''}}

\emph{4'33''}由美国实验作曲家约翰·凯奇于 1952 年创作,适用于任何乐器或乐器组合。\emph{4'33''}的乐谱采用传统记谱法,三个乐章没有刻意表示的声音。钢琴家上台,坐在钢琴前,合上钢琴盖,标志着第一乐章的开始。然后,钢琴家打开钢琴盖,再次合上,表示下一个乐章。如此重复一次。钢琴盖的闭合和打开并不是为了发出声音,而是作为乐曲时间结构的提示,将静止划分为不同的部分。\cite{4'33''}

要讨论凯奇的音乐的独创性,首先应从当时西方音乐的背景谈起。20 世纪 30 年代,西方音乐传统的主导思想是:音乐是人类组织起来的声音。但是,凯奇在洛杉矶短暂地跟随阿诺德·勋学习后,提出了一个激进的观点:任何声音都可以是音乐,音乐不需要人类的意图。当时,实验艺术开始挑战各学科的现状,与视觉艺术中的抽象表现主义和文学中的垮掉的诗歌并驾齐驱。

凯奇在创作这首曲子时,受到了禅宗、铃木大拙学说的和中国的老庄思想的影响。禅宗和铃木大拙学说强调“正念”和“包容”环境各个方面的重要性,同时,他也推崇《老子》的“大音希声”的音乐审美观。禅宗中“空”的概念与\emph{4'33''}尤其相关。在这里,“空”并非指虚空,而是指万物相互联系的潜在空间。凯奇将寂静诠释为环境声音的画布,反映了这一空性概念:寂静不是没有声音,而是有机会感知无意声音的充实。音乐不是一定要达到某种目的,它可以是无意义的存在,可以有声,也可以无声,而且不管是传统的音乐性的还是其他形式的,所有的音响都是可以接受的。

\emph{4'33''}最深刻的方面是当属它给我带来的体验。\emph{4'33''}摒弃了预期的音乐内容,而给了听众倾听周围声音的空间--衣服的沙沙声、其他观众的的呼吸声、远处通风系统的嗡嗡声,以及从表演空间外传入的更加难以预测的声音等。这首曲子成为对生活声音本身的探索,通过其无声的乐谱,向我们传达了一种新的听觉体验。没有声音的乐曲让我们重新思考声音的本质,以及我们如何感知和解释声音。它让我们思考音乐的定义,音乐与噪音之间的界限,并最终思考艺术的本质和基本原则。

事实上,凯奇的\emph{4'33''}也与 20 世纪艺术运动相吻合。通过其无声的乐谱,该作品成为一个强有力的声明,邀请我们重新体验世界,不带偏见地倾听,并在每天围绕我们的意想不到和被忽视的声音中进行思考,并寻找和谐。

\section{\emph{Gesang der Jünglinge}}

\emph{Gesang der Jünglinge}由卡尔海因兹·施托克豪森(Karlheinz Stockhausen)于 1955 年至 1956 年间创作,是电子音乐领域的里程碑式作品,代表了传统基督教宗教主题与前卫电子音效的融合。它是最早将电子合成声音与人声完美融合的作品之一。斯托克豪森是二战后欧洲先锋派的一员,这个团体在战后努力重新定义音乐。他们寻求新的声音、新的结构和新的音乐思维方式,而施托克豪森对这首乐曲的贡献正是这些努力的核心。\cite{Gesang}

在\emph{Gesang der Jünglinge}中,施托克豪森开创性地使用了电子音效。他录制了一位男高音(男童)歌唱圣经的声音,然后通过剪切、拼接和各种电子效果(如反转、变速和滤波)对其进行电子处理。少年歌唱的碎片使得空间很独特和庞大,让听者进入独特的奇幻氛围。旋律性极强的人声使得其音响效果在复杂的电子音响音色之中有有一丝独特的单纯。而复杂的正弦波,脉冲群,白噪,泛音并没有使得这首作品变得混乱无章,反而是更加具有代表性。多声道的创作又让音乐充满空间感。这首作品的声音效果非常丰富,但是又不会让人感到混乱,而是让人感到非常的和谐。

在欣赏这一首曲子时,我重新思考音乐的本质以及人类和合成音乐在音乐创作中的作用。在我看来,这首乐曲有力地证明了音乐不只是旋律、节奏和音高的堆砌,更是声音参数的深思熟虑和探索。人类的智慧能够开辟新的艺术道路,并在人类与技术的融合中发现美感和表现力。事实上,\emph{Gesang der Jünglinge}这种精心设计的声音结构,要求我们在接受时开展积极的思考过程,重新定义了音乐的边界和可能性。

\section{总结}

对凯奇的\emph{4'33''}来说,凯奇的作品将音乐重新定义为一种语境,而非刻意构建的声音。在这个例子中,“思想”是一种意识、心智和对假设的质疑。凯奇利用结构化声音的缺失,让听众意识到他们周围一直存在的、却往往被忽视的声音。这是一种深刻的“用声音思考”,声音不是被创造出来的,而是被认可的。因此,\emph{4'33''}中的 “音乐”是一个认知过程,是对周围听觉环境的积极参与,也是对音乐和声音构成要素的反思。

而对于施托克豪森的\emph{Gesang der Jünglinge},相比之下,施托克豪森的作品既使用了人类声音,也使用了合成声音。在这里,施托克豪森的通过调整音乐参数的序列组织和声音的空间分布,从而产生了一种既具有反思性又具有创新性的复杂音景。从声音的组织、处理和呈现方式中,他提供了一种全新的方式来感知人类发声与电子声音之间的关系,并让我们思考声音的本质。

在这两首作品中,作曲家将声音作为表达复杂思想的主要工具。他们邀请听众与他们一起思考,在听觉体验中解释和寻找意义。在这些情况下,音乐成为作曲家、表演环境和听众之间的哲学对话,而声音则是这种互动的语言。“音乐是用声音思考的艺术 ”,事实上,音乐是一种智力活动,是作曲家和听众探索和创造声音这一无形材料的领域,可以像任何其他形式的艺术一样,给人以强烈的情感冲击和严谨的智力体验。它表明,音乐不仅仅是音高、节奏和速度的组合,而是传递思考的。正如凯奇和施托克豪森的曲子,他们用声音提出问题,探索可能性,最终改变我们对音乐的认识。

% plain(参考文献的条目编号是按照字母的顺序)
% unsrt(参考文献的条目编号是按照引用的顺序)
% alpha(参考文献的条目编号是按照作者名字和出版年份的顺序)
% abbrv(缩写格式)

\bibliographystyle{unsrt}
\bibliography{cite.bib}

\end{document}