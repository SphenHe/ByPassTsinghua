\documentclass{article}
\usepackage[UTF8]{ctex}

\usepackage{amsmath}        %数学公式
\usepackage{amssymb}
\usepackage{cite}           %引用

\usepackage{graphicx}       %插入图片
\usepackage{float}          %设置图片浮动位置
\usepackage{subfigure}      %插入多图时用子图显示

\usepackage{listings}
\usepackage{xcolor}

\usepackage{fancyhdr}       %页眉、页脚、页码
\usepackage[a4paper, margin=1in]{geometry}    %纸张大小
\usepackage{longtable}

\setlength{\headheight}{16pt}
\pagestyle{fancy}
\fancyhf{}

\pagenumbering{arabic}

\begin{document}

\fancyhead[L]{Jerry}
\fancyhead[C]{核辐射上半学期口试}
\fancyfoot[C]{\thepage}

\section*{第一章}

\emph{1. 稳定曲线为什么刚开始与 Z=N 的直线重合?为什么后来又偏离了这个直线,向哪个方向偏离?$\beta$稳定曲线会不会
向高 A 区无限延伸?}

在刚开始的一段,中子和质子分别计算能级,且两者能级之间能量差不是很大,于是在Z=N的时候粒子能量最低,最稳定;在A(Z)逐渐增加的时候,粒子的电荷排斥逐渐增大,使得偏向N轴;不会向高A区无限延伸,因为在高A区随着A的增加,粒子的比结合能降低会越来越快,使得进行$\alpha$衰变的衰变能增大,促使$\alpha$衰变的发生

\emph{}

\emph{2. 如何由“原子核有确定的宇称”,推出“其电偶极矩必然为 0”的结论?对于电四极矩可以有如此明确的断言吗?}

原子核有固定的宇称,则原子核的波函数是对称分布的。对电偶极矩中的r项,使用勒让德展开,可以发现偶极电势是奇对称的,从而电偶级距必然为0,与物理含义的原子核对称也相符合;对于电四级距,因为这一项不是奇对称的,且这一项物理含义是着原子核的形状(长椭球/球/扁椭球),从而不能有明确的断言。

\emph{}

\emph{3. 质量过剩描述的对象是谁?辨析它和质量亏损这个概念的区别。}

质量过剩是指原子的质量,是以$^{12}C$原子的1/12为基准的一个方便计算的量;质量亏损是指所有核子质量之和减去原子核质量,是一个可以用于衡量原子核稳定性和体现原子核质量的量。

\emph{}

\emph{4. 随着核子数 A 的增大,液滴模型中几个比结合能项会分别怎么变化?液滴模型有什么不足之处吗?}

$$\varepsilon=a_V-\frac{a_S}{A^{1/3}}-\frac{a_C Z^2}{A^{4/3}}-\frac{a_{sym}(A/2-Z)^2}{A^2}+\frac{\delta a_p}{A^{3/2}}$$

假定Z不变,随着核子数A的增加,液滴模型中的体积能不会变化,表面能会按照$A^{-1/3}$减少,库仑能会按照$A^{-4/3}$减少,对称能会减少,对能会根据核子数量的奇偶性变化。(如果不假设Z不变,则应有库伦能随着A增大而增大,因为$Z^2$是次数比$A^{4/3}$高,同时对称能也会相对变大,因为后期Z和N相差越来越大)。不足之处体现为这个模型无法解释“幻数”的存在。

\emph{}

\emph{5. 原子核从基态来到某个激发态时,统计性、宇称、磁矩、电四极矩会发生变化吗,为什么?}

从基态到激发态时,原子核统计性只与原子核的核子个数有关,不变化;不同能级角动量不同,于是宇称会发生变化;磁矩也会随着角动量的变化发生变化;原子核激发后形状会发生变化,改变电荷的分布,从而改变电四极矩

\emph{}

\emph{6. 如何理解核力的“自旋-轨道耦合”项对于壳层结构幻数的意义?}

核力的“自旋-轨道耦合”项是核的壳层模型的最后一块拼图,成功的解释了大于20的幻数的存在

\emph{}

\section*{第二章}

\emph{1. 在什么情况下,原子核被探测器测量出的半衰期和自己实际的半衰期是不同的?如何快速估计放射平衡的达成时间?}

高速运动的核子,由于相对论的钟慢效应,测出的半衰期比实际的偏长;放射平衡一般在5-7个半衰期达成平衡

\emph{}

\emph{2. 放射源的制备问题}

\emph{ 2.a. 影响其活度的 5 个因素分别是什么?}

\emph{ 2.b. 为了高的活度,无限地延长照射时间,是否合理?}

\emph{ 2.c. 若为了获得最大活度的 75\%,需要照射多久?}

5个因素分别为:1. 靶核A的数量$N_{target}$,无量纲数量;2. n与A的反应截面$\sigma_0$,单位为cm$^2$;3. 中子注量率$\Phi$,单位为1/(cm$^2$s);4. 余核B的衰变常数$\lambda$,单位为1/s;5. 照射时间$t$,单位为s。截面和入射核,余核,和入射核子的相对运动动能有关。

不合理,应该考虑经济效益,实际上在4个半衰期之后活度就已经到达最大活度的90\%以上了,一般能满足需求。

活度的公式为$A=A_0(1-e^{-\lambda t})=A_0(1-2^{-t/T_{1/2}})$,容易计算得出两个半衰期后即可得到75\%活度的源

\emph{}

\emph{3. 说出至少两种测量核素半衰期的方法?(扩展阅读第二章阅读材料 2)}

from hhw()

方法一:直接观察放射性强度随着时间的指数衰变。

方法二:对于长半衰期,可以直接使用,测量衰变强度,然后通过化学方法测量N。

方法三:对于超短半衰期,可以使用延迟重合技术,把两个相近的衰变作为重合分析。

\emph{}

\emph{4. 工作中常用的 $\gamma$ 源(如 137Cs 或 60Co)不大可能是单纯的 $\gamma$ 源,而通常也是个 $\beta$ 源,为什么?(可结合第三章内容)}

因为$\gamma$衰变的半衰期一般较短,最长的也在小时的量级(同质异能态很少见),很快衰变完而不能用于工作中。所以一般采用“母牛“的形式,使用半衰期较长$\beta$和$\gamma$衰变从而达到长期/暂时平衡,使得源能够持续放出$\gamma$射线。所以一般通常也是一个$\beta$源

\emph{}

\section*{第三章}

\emph{1. 针对 $\alpha$ 衰变、$\beta$ 衰变和 $\gamma$ 跃迁,解释衰变能、角动量、宇称这三个量对其衰变过程的影响。}

对$\alpha$衰变,要求衰变能E>0;对$\beta$衰变,如果是$\beta^-$衰变,需要满足反应大于0,如果是$\beta^+$衰变,需要满足反应大于2$m_ec^2$,如果是EC衰变,需要满足反应大于$\varepsilon_{k,l,m}/c^2$;对$\gamma$跃迁,只需要满足选择定则就可以

角动量宇称的决定即为选择定则

\emph{}

\emph{2. 试着定性说明为什么在 $\alpha$ 衰变中,衰变能一般在约 4-9MeV 之间,既不会很大,也不会很小?}

如果太小就难以突破势垒,导致反应无法发生;在原子核中难以让 $\alpha$ 粒子获得很大的能量,所以衰变能一般不会太大

\emph{}

\emph{3. 讨论一下在 $\alpha$ 衰变、$\beta$ 衰变和 $\gamma$ 跃迁过程可能产生的粒子(提示:各衰变后续过程产生的粒子也应尽可能考虑)。}

\emph{ 3.a. 不带电的粒子}

\emph{ 3.b. 带电的粒子}

\emph{ 3.c. 能量取分立值的粒子}

\emph{ 3.d. 能量呈现连续分布的粒子}

在$\alpha$衰变中,衰变本身会放出分立能量的子核和$\alpha$粒子,子核可能有激发态,在退激发的时候可能发生$\gamma$衰变,从而放出能量分立的$\gamma$光子,后续可能放出能量分立的$\gamma$光子和内转换电子。

在$\beta$衰变中,$\beta^+$衰变会放出能量为连续值的正电子、中微子,$\beta^-$会放出能量为连续值的负电子、反中微子,并放出能量连续的$\gamma$光子和内转换电子。对于EC衰变,会放出能量分立的能量分立的俄歇电子、特征X射线。

在$\gamma$衰变中,会放出能量分立和$\gamma$光子,后续可能放出能量分立的$\gamma$光子和内转换电子。

\emph{}

\emph{4. 跃迁矩阵元对于 $\beta$ 衰变的衰变常数影响很大的原因是什么?}

矩阵跃迁元的大小直接决定了反应的级次,矩阵跃迁元越小,反应级次越低,反应就越容易发生。

\emph{}

\emph{5. 请解释一下$\beta$衰变的选择定则的形成过程?}

$\beta$衰变可能放出两种轻子,根据轻子组的自旋角动量方向,可以带走0或1个角动量,出现了F跃迁和G-T跃迁;然后根据宇称守恒判断属于哪一级的衰变;同时,带走的轨道角动量越小的越容易发生

\emph{}

\emph{6. 请解释一下$\gamma$跃迁的选择定则的形成过程?}

首先,$\gamma$跃迁中,光子带走的角动量不能是0;其次,然后根据角动量守恒和宇称守恒,通过初态和末态原子核的状态,判断这个$\gamma$跃迁属于的级次和能发生的跃迁类型;同时,有电跃迁比磁跃迁更容易发生

\emph{}

\emph{7. 既然原子核的电偶极矩必然为 0,为什么$\gamma$跃迁的电偶极跃迁仍可能是存在的,而且(如果存在的话)是最强的?}

电偶级距是描述的原子核本身的性质。电跃迁与磁跃迁描述的是原子核内部的变化,所以电偶极跃迁是存在的。当电偶极跃迁存在时,它带走的角动量最小,而且电跃迁比磁跃迁容易发生,所以E1是最容易发生的,是最强的跃迁。

\emph{}

\emph{8. 把(由 Z 个质子、N 个中子构成的)原子核的每一个能级都用一条线画出来,线的宽度代表了能级宽度 $\gamma$。}

\emph{ 8.a. 如果只允许其中一条线的宽度为 0,则这条线会对应哪个能级?}

\emph{ 8.b. 对于该原子核,a所述的宽度为 0 的线一定存在吗,为什么?}

\emph{ 8.c. 一般来说,能级越高,线的宽度是越大还是越小,为什么?}

这个能级为基态能级,只有基态能级是稳定的

不一定存在,如果原子核不稳定,那么基态能级也会有宽度,那就没有宽度为0的线

能级越高,原子核的能量越大,线的宽度就越大

\end{document}
