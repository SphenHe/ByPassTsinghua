\documentclass{article}
\usepackage[UTF8]{ctex}

\usepackage{amsmath}        %数学公式
\usepackage{amssymb}
\usepackage{cite}           %引用

\usepackage{graphicx}       %插入图片
\usepackage{float}          %设置图片浮动位置
\usepackage{subfigure}      %插入多图时用子图显示

\usepackage{listings}
\usepackage{xcolor}

\usepackage{fancyhdr}       %页眉、页脚、页码
\usepackage[a4paper, margin=1in]{geometry}    %纸张大小
\usepackage{longtable}

\usepackage[hyphens]{url}
\usepackage{hyperref}       %超链接

\hypersetup{
    hidelinks,
    colorlinks=true,
    allcolors=black,
        pdfstartview=Fit,
        breaklinks=true
}

\setlength{\headheight}{16pt}
\pagestyle{fancy}
\fancyhf{}

\pagenumbering{arabic}

\begin{document}

\fancyhead[L]{Jerry}
\fancyhead[C]{核辐射下半学期口试}
\fancyfoot[C]{\thepage}

\tableofcontents
\newpage

\section{问题1:形成全能峰的最后一个反应是什么?}
\label{sec:prob1}

\textbf{问题描述:} 在测量 $\gamma$ 能谱时,我们希望 $\gamma$ 光子的全部能量都可以用来形成信号,然后在“全能峰”(全部光子能量的峰)那里形成贡献。如果,你现在面对的是一个 1.33MeV 的光子,并且最后在探测器输出的能谱中确实也看到了它的全能峰,请分析一下该全能峰可能的形成过程。解释一下:最后一步反应是谁?

\textbf{解答:} 形成过程:

$\gamma$ 光子可以:1.通过光电效应形成光电子和俄歇电子;2.通过康普顿散射生成反冲电子;3.通过电子对效应生成正负电子,来将光子能量沉积,转换为次级电子。次级电子将通电离或者激发产生载流子。

载流子在气体探测器中是电子离子对,在闪烁体中是电子空穴对-荧光电子-光电子,在半导体探测器中是电子空穴对。次级电子也可能发生韧致辐射,产生的X射线发生光电效应产生光电子。最后通过探测载流子的数目得到全能峰。

最后一步是光电效应,因为康普顿散射无法沉积所有的能量。除此之外,也有可能是电子对效应?($\gamma$光子能量恰好略大于1.022keV时)。

\section{问题2:随机性是如何“演变”为确定性的?}
\label{sec:prob2}

\textbf{问题描述:} 核辐射测量过程的各个环节都包含有随机性。例如:我们在测量活度时,放射源中每个原子核的衰变都是独立而随机的,那我们为什么又可以言之凿凿地说“测得某源的活度是多少多少”呢?又如:带电粒子在探测器介质内的电离过程是随机的,产生的载流子数目并不确定,每个载流子的产生都是随机的,可是为什么我们最终又能够以很好的分辨率(如半导体探测器)来形成全能峰呢?简而言之,微观过程总是难以免除随机性,那么宏观过程是如何变得越来越确定呢?

\textbf{解答:} 首先随机性是满足一定的概率分布的,在数量够多的时候,由中心极限定理可以得出这就是期望值。其次,无论是什么分布,都有 $\nu \propto \frac{1}{\sqrt{N}}$ ,即满足“越多越好”的原则,在数量足够大的时候可以使得随机性导致的测量误差远小于真实值,从而使得测量结果越来越确定,分辨率更好。而增大数量的方式有很多,如雪崩,如光电倍增管,又如半导体探测器的减少激发所需的能量。

\section{问题3:关于湮没辐射的问题?}
\label{sec:prob3}

\textbf{问题描述:} 在 $\gamma$ 能谱中,我们通常可以看到 511keV 的湮没峰,它来自于一个正电子和一个负电子湮没后的 2 个 511keV$\gamma$ 光子中的 1 个,那么,我们为什么几乎看不到 1.022MeV 的“湮没峰”呢?此外,看到湮没峰时,你认为正电子的产生方式可能是什么?

\textbf{解答:} 511keV的湮没峰是由探测器外的正负电子对湮没形成的,由于湮没产生的两个$\gamma$光子方向相反,最多只会有一个光子进入探测器。正电子可以由电子对效应产生,也可以由$\beta+$衰变产生。

\section*{问题4:选做:为什么基于 MeV X-ray 的集装箱检测系统只能测量质量厚度信息?}
\label{sec:prob4}

\textbf{问题描述:} 在集装箱检测系统中,通常会用 6MeV 的电子与钨靶来制备轫致辐射,然后根据 6MV X-ray 的衰减量来分析集装箱内所藏匿物质的情况。但是基于 6MV X-ray 的透射成像技术,只能提取射线穿透路径上的集装箱内物体的质量厚度,请问这是为什么?

\textbf{解答:} 对于6MeV的X-ray,康普顿散射占优,(成像原理是基于原子与X-ray的的康普顿散射)在康普顿散射中,反应截面正比于Z,因而测量时可以且只能得到质量密度。

\section{问题5:为什么实测计数一定低于期望计数?}
\label{sec:prob5}

\textbf{问题描述:} 当我们测量某个放射源(例如 137Cs)的活度时,若根据源强和探测器的探测效率算出期望计数率为 m,但实测计数率的期望值 n 总是比 m 小,请问这是为什么?

\textbf{解答:} 因为探测器具有处理信号的时间(死时间),在死时间入射的粒子不会被记录。所以计数率小于真实值。对于一个活度为$m$的源,在处理的时候,由于到达的概率为$m e^{-mt}$,可以推出实际计数率$n$与真实值$m$的关系为$n=me^{-m\tau}$,其中$\tau$是分辨时间。

\section{问题6:和峰的成因?}
\label{sec:prob6}

\textbf{问题描述:} 用 NaI(Tl)探测器测某单能 $\gamma$ 源,不妨认为是 137Cs 的 0.662MeV,通常,我们不仅能够看到 0.662MeV 的全能峰,还有可能看到 1.324MeV 的和峰,这是为什么?和峰是否可以完全消除呢?

\textbf{解答:} 当两个$\gamma$光子恰好一同入射,被当作是一个$\gamma$光子,从而测得的能量是两个原$\gamma$光子的和。不可以完全消除,只能减弱和峰的影响。

\section*{问题7:选做:如何利用放射源来制备时钟?}
\label{sec:prob7}

\textbf{问题描述:} 若你拥有活度已知(大小可以“随心所欲”地设定)的放射源,以及本征效率已知、死时间很小(可忽略)的探测器和计数器,是否可以制备出一个比较精准的秒表来?

\textbf{解答:} 选择 $^{226}$Ra 作为放射源,使用高纯锗探测器。1. 将 $^{226}$Ra 放射源置于探测器前,并记录初始计数率 $N_0$。2. 经过一段时间 t 后,记录计数率 $N(t)$。3. 根据放射性活度的衰变规律,计算时间间隔 $t$:

\section{问题8:为什么气体探测器通常具有较好的 n/$\gamma$ 区分能力?}
\label{sec:prob8}

\textbf{问题描述:} 气体探测器在测量中子和 $\gamma$ 时,信号表现很不一样,前者幅度大,后者幅度小,这使得利用甄别阈来区分中子和 $\gamma$ 变得很容易,算是气体探测器的天然优点,请问这是为什么?

\textbf{解答:} 因为与$\gamma$光子的反应概率为$-1e^{-N\sigma D}$,很难沉积大部分能量,且在1.5MeV的能量时能量沉积能力较弱。因此$\gamma$光子能量沉积会和中子的能量沉积有很大的不同。

\section{问题9:探测器的电流形状是确定的吗?}
\label{sec:prob9}

\textbf{问题描述:} 从气体探测器开始,我们会学习 3 种探测器(气体探测器又可以分为:普通电离室、圆柱形电离室、屏栅电离室、正比计数器、G-M 管)。这些探测器都会将射线沉积在灵敏体积内的能量转换为电流。提问:

1) 电流的形状(不是幅度)是确定的吗?对哪些探测器可以认为是确定的,对哪些则不能认为是确定的?

2) 电流的形状若不定,则会对射线能量的分析造成什么影响?我们又是如何解决这个问题的?

\textbf{解答:}

\begin{enumerate}
    \item 对平板(普通电离室),形状和幅度都不是确定的,使得输出脉冲幅度将不再正比于射线的沉积能量。也许可以调整R0C0使其远大于电子运动时间,切换到累计模式来减少影响。更常见的方式是采用圆柱形电离室。
    \item 对圆柱形电离室,形状不是确定的,利用圆柱形电场的特点来减少入射粒子位置的变化,b/a既可使得电子漂移距离几乎相同,此时可以读取沉积的能量。
    \item 对屏栅电离室,形状不是确定的,轨迹不同到达的时间不一样使得电流形状不一样。
    \item 对正比计数器,形状是确定的。%雪崩
    \item 对G-M管,这玩意形状不确定,由于放大倍数不定,不能测能量。
    \item 对闪烁体探测器,形状是确定的。%可以调整R0C0来改变工作模式。
    \item 对半导体探测器,形状不是确定的。可以通过在输出回路加一个电荷灵敏前置放大器(大电阻小电容,使得RfCf=R0C0大且输出噪声小)来解决。
\end{enumerate}

\section{问题10:什么是等效电路?}
\label{sec:prob10}

\textbf{问题描述:} 进入探测器的学习阶段之后,我们就免不了老提“等效电路”这个术语,那么构成等效电路的电阻、电容是哪些呢?这个电路有个冲击响应,可以由什么函数来描述呢?这个冲击响应又为什么会对一个探测器是工作在“脉冲”模式,还是工作在“累计”模式构成影响呢?

\textbf{解答:} 构成等效电路的电阻有导线电阻接触电阻等、电容有探测器本身的电容,导线电容,电荷积累的电容等;冲击响应是一个指数函数($\frac{e}{C_0}e^{-t/R_0C_0}$),在调整外回路的R0C0后可以使得这个卷积核改变,从而决定是脉冲模式/累计模式。

\section{问题11:为什么正比计数器的输出信号幅度正比于射线沉积的能量?}
\label{sec:prob11}

\textbf{问题描述:} 在正比计数器中,我们用的 R0C0 并不大,也许只有几个 μs,而又知道正比计数器中是离子在做主要贡献,后者的弹道亏损问题应该是不能忽略的,那么为什么利用正比计数器还能测量射线的沉积能量——即 V 和 Edep 之间还存在正比关系呢?

\textbf{解答:} 在正比计数器中,由于电离位置几乎是相同的,所以弹道亏损几乎也是相同的,这也使得电流形状几乎是相同的;电流面积受到的影响因素N0正比于Edep,倍增系数只有工作电压V0决定。这使得V正比于Edep。

\section{问题12:气体探测器 vs 闪烁探测器的 $\gamma$ 探测效率?}
\label{sec:prob12}

\textbf{问题描述:} 气体探测器的 $\gamma$ 探测效率通常不高,闪烁探测器的则可以很高,是什么原因导致了前者低、后者高呢?

\textbf{解答:} 同问题\ref{sec:prob8},气体与$\gamma$的反应截面太小了,而闪烁体可以选择Z大的材料(光电效应反应截面正比于Z的5次方),使得反应截面增加,探测效率高。

\section{问题13:为什么在测量能谱时,无机闪烁体的弹道亏损可以不考虑?}
\label{sec:prob13}

\textbf{问题描述:} 我们用无机闪烁体可以测量射线的沉积能量,此时一般 R0C0 选的和无机闪烁体的发光时间相当(对此你 可以理解为电流持续时间和等效电路的时间常数是一样的),因此弹道亏损一定是存在的。但是,我们似乎并不担心由此而导致的能量测量问题(即输出信号的幅度和能量沉积之间的正比关系),为什么?

\textbf{解答:} 参考问题\ref{sec:prob9},弹道亏损程度是固定的,所以Vmax正比于Q

\section{问题14:为什么闪烁探测器的能量分辨率通常最差?}
\label{sec:prob14}

\textbf{问题描述:} 众所周知,闪烁体探测器的探测效率可以说是最好的,但是它的能量分辨率却排在半导体、气体之后,几乎是最差的,这是为什么?

\textbf{解答:} 因为:1.闪烁体探测器有自吸收问题,而且荧光光子被输运到光阴极的概率也不相同,因此能量相同而位置不同的能量沉积事件,到达光阴极的荧光光子数目不同;2.荧光光子不是载流子,到光阴极转换为电子,是一个10\%的低效率过程,损失了数目。这两个过程损失了数目,使得能量分辨率变差,

\section{问题15:PN 结对于 Si 探测器的意义是什么?}
\label{sec:prob15}

\textbf{问题描述:} 在金硅面垒探测器中,PN 结的作用是巨大的,如果没有 PN 结,半导体探测器能量分辨率好的优点就不可能发挥出来,这其中具体的道理是什么呢?

\textbf{解答:} PN结清扫了载流子,使得结区的电阻率增大,使得漏电流和结电容小,对探测的影响小。

\section{问题16:相较于金硅面垒探测器,高纯锗探测器的效率为什么可以很高?}
\label{sec:prob16}

\textbf{问题描述:} 金硅面垒探测器通常只适合于测量重带电粒子和低能电子,对于高能 $\gamma$ 射线其效率很低。但是,同样作为 PN 结型探测器的高纯锗,却可以实现很高的 $\gamma$ 探测效率,道理是什么?

\textbf{解答:} 高纯Ge的杂质浓度极低,相应的电阻率很高;空间电荷密度很小,P区的耗尽层厚度大

\section{问题17:为什么半导体探测器的前置放大器的 R0C0 很大?}
\label{sec:prob17}

\textbf{问题描述:} 在半导体探测器中,前置放大器的时间常数 R0C0 通常会选择为 ms 这么大,这是为什么?如果不这样选择,可能会导致什么问题?

\textbf{解答:} 使得R0C0 > > 电流持续时间(10ns),使得弹道亏损减少。弹道亏损变大,能量分辨率降低

\section{问题18:哪种探测器的等效电容是不确定的、而是随工作条件变化的?}
\label{sec:prob18}

\textbf{问题描述:} 在三种探测器中,等效电路中的 C0 都可以被认为是由探测器自身的 C1,分布电容 C',仪器的输入电容 Ci 共同构成的,看上去应该是确定不变的。但是有一种探测器,它的 C1 是变的,导致了 C0 也就变了。这是哪种探测器,C1 改变的原因是什么?

\textbf{解答:} 半导体探测器,工作电压会影响结区宽度,进而改变探测器电容

\section{问题19:请列举出三种探测器在探测过程中所涉及的级联过程。}
\label{sec:prob19}

\textbf{问题描述:} 在第七章的学习中,我们知道了大的数量往往意味着好的统计性(即相对涨落小),但这个结论对于级联变量并不成立,因为级联变量中的第二级过程虽然增加了数量,却实际上恶化了统计性(相对涨落增大)。请你回顾下气体、闪烁和半导体探测器,列举出在这些探测器中的哪些过程涉及到了级联变量?

\textbf{解答:}

\begin{enumerate}
    \item 气体探测器:雪崩过程
    \item 闪烁体探测器:光电倍增管PMT
    \item 半导体探测器:电荷灵敏前置放大器
\end{enumerate}

\section{问题20:死时间校正的问题?}
\label{sec:prob20}

\textbf{问题描述:} 在一个用来测量核计数的探测器系统中,死时间是否必然存在?为什么?如果用 NaI(Tl)探测器去测量某 $\gamma$ 源的计数,若全能峰的计数率为 10000cps,峰总比是 0.5,死时间为 10μs,则死时间校正因子是 1/0.9,对吗?如果有错,错在哪里?

\textbf{解答:} 不对,首先是应该是通过总计数率来计算校正因子,其次应该考虑计数率过大产生的解;正确的方案:$x e^{-\frac{x}{0.5}\tau}=10000cps$,解得x=12958.6或127132。

\section{问题21:真符合与偶然符合?}
\label{sec:prob21}

\textbf{问题描述:} 真符合发生的条件是什么?偶然符合能被完全消除吗?请分别解释其原因。

\textbf{解答:} 真符合发生的条件是两个事件物理上相关;偶然符合不能被完全消除,因为偶发情况总会出现。

\section{问题22:一个 $\gamma$ 能谱里面有多少个可能的峰?}
\label{sec:prob22}

\textbf{问题描述:} 在 $\gamma$ 能谱里面,你可能看到很多峰,都有什么峰呢?它们的成因分别是什么?

\textbf{解答:}

\begin{enumerate}
    \item 反散射峰,环境康普顿
    \item 湮灭峰,电子湮灭且只可进入一个,见\ref{sec:prob3}
    \item 单逃逸峰
    \item 双逃逸峰
    \item 康普顿边缘,康普顿能沉积的最大能量就这么点
    \item 光电峰,光电效应还要吸收一点能量导致的
    \item 全能峰
    \item 和峰,两个混一起了
\end{enumerate}

\section{问题23:一个探测器的能谱形状,与计数率有关吗?}
\label{sec:prob23}

\textbf{问题描述:} 我们用 NaI(Tl)探测器来测量某 $\gamma$ 射线源的能谱时,是否可以认为,所测 $\gamma$ 能谱的形状,与源强是无关的?为什么?

\textbf{解答:} 不可以,和峰会受到源强的影响,源强的时候和峰会增加

\section{问题24:在 $\gamma$ 能谱的特征中,有哪些是与“逃逸现象”有关的?}
\label{sec:prob24}

\textbf{问题描述:} 在测量 $\gamma$ 能谱时,我们会遇到很多种“逃逸现象”,请尽可能多地列举出由光子逃逸所致的能谱特征。

\textbf{解答:}

\begin{enumerate}
    \item 光电峰
    \item 多次康普顿散射
    \item 康普顿坪
    \item 康普顿边缘
    \item 单逃逸峰
    \item 双逃逸峰
\end{enumerate}

\section{问题25:如何测量慢中子?}
\label{sec:prob25}

\textbf{问题描述:} 在测量能量很低的慢中子时,例如 1eV 的中子,应该利用什么方法?可否用反冲法,为什么?

\textbf{解答:} 应该使用辐射俘获/产生带电粒子的核反应(n,$\gamma$)/(n,c);不可以使用反冲法,能量<10MeV的中子入射时,反冲质子能谱分布可认为是均匀分布/中子的能量变化太小不便于测量。

\section{问题26:中子灵敏度的物理意义是什么?}
\label{sec:prob26}

\textbf{问题描述:} 在测量慢中子时,有个中子灵敏度的概念,请问它的量纲是什么?怎么理解它的物理意义?

\textbf{解答:} $cm^2$,理解为中子反应截面?

\section{问题27:如何测量快中子?}
\label{sec:prob27}

\textbf{问题描述:} 对于数 MeV 的快中子,通常应该怎么测?为什么不能沿用慢中子的测量方法了?

\textbf{解答:}

\begin{enumerate}
    \item 核反应法,反应能> >中子能量,对于低能中子能量的测量是很困难的/只能测量注量不能测量能量
    \item 核反冲法,太慢了不方便探测(应该是这个)
    \item 核裂变法,许多重核只有当中子能量大于“阈值”时“才”发生裂变/只能测量注量不能测量能量
    \item 活化法/只能测量注量不能测量能量
\end{enumerate}

\section*{问题28:选做:中子的存活时间及分布函数}
\label{sec:prob28}

\textbf{问题描述:} 在 F.Reines 和 C.L. Cowan 开展的反电子中微子测量实验中,利用反电子中微子和质子产生了正电子和中子,通过正电子事件和中子事件的符合,确认了反电子中微子的存在。在这个过程中,正电子湮没得快,中子最终被 113Cd 俘获得慢,请分析一下,一个 1MeV 的中子在掺有一定浓度 113Cd 的溶液中的存活时间服从什么样的分布,平均寿命由什么决定?为了回答这个问题,需要用到第四章、第九章和第十三章的知识,大家需要先回顾课程文件中的第三章阅读材料(14) 。

\textbf{解答:} 不会,摆了

\end{document}
