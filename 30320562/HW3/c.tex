\documentclass[]{article}
\usepackage{ctex}
\usepackage{amsmath}        %数学公式
\usepackage{cases}          %联立编号
\usepackage{cite}           %引用
% \usepackage{enumitem}       %编号

\usepackage{graphicx}       %插入图片
\usepackage{float}          %设置图片浮动位置
\usepackage{subfigure}      %插入多图时用子图显示

\usepackage{anyfontsize}    %解决一个奇怪的字体大小报错问题
\usepackage{fancyhdr}       %页眉、页脚、页码
\usepackage[a4paper, margin=1in]{geometry}    %纸张大小

\newcommand\f[2]{\frac{#1}{#2}}
\newcommand\pf[2]{\frac{\partial#1}{\partial#2}}
\newcommand\df[2]{\dfrac{#1}{#2}}
\newcommand\pdf[2]{\dfrac{\partial#1}{\partial#2}}
\newcommand\zsin[1]{\frac{e^{i#1}-e^{-i#1}}{2i}}
\newcommand\zdsin[1]{\dfrac{e^{i#1}-e^{-i#1}}{2i}}
\newcommand\zcos[1]{\frac{e^{i#1}+e^{-i#1}}{2i}}
\newcommand\zdcos[1]{\dfrac{e^{i#1}+e^{-i#1}}{2i}}
\newcommand\zline[1]{#1-\overline{#1}}

\newcommand\dg[2]{#1^{\circ}#2'}

\setlength{\headheight}{16pt}
\pagestyle{fancy}
\fancyhf{}


\title{决策方法论 - HW3}
\author{\LaTeX\ by\ Jerry\ }
\date{\today}
\pagenumbering{arabic}

\begin{document}

\fancyhead[L]{Jerry}
\fancyhead[C]{决策方法论 - HW3}
\fancyfoot[C]{\thepage}

\section{试解释似然性(Likelihood)与概率(Probability)的区别}

似然性(Likelihood)和概率(Probability)这两个概念在统计学中都非常重要,但它们指的是不同的概念,适用于不同的情况。

概率描述的是在已知一些参数的情况下,一个或多个事件发生的可能性。它是对未来事件发生的不确定性的度量。比如,如果我们知道一枚公平的硬币,那么我们可以说它正面朝上的概率是50\%。

而似然性是在给定某些数据时,不同参数值的支持度或合理性的度量。它不是对事件发生的预测,而是一种在已经观察到事件之后,评估不同参数情况可能性的方式。如果我们观察到一枚硬币扔了100次,其中正面朝上70次,似然函数将帮助我们评估这是一枚公平硬币的可能性。

从数学角度来看,概率是在给定参数的情况下,数据的函数;而似然性是在给定数据的情况下,参数的函数。即:

\begin{itemize}
    \item 概率(Probability):$P(Data | Parameters)$,给定参数求数据的概率。
    \item 似然性(Likelihood):$L(Parameters | Data)$,给定数据求参数的似然性。
\end{itemize}

因此,似然性和概率虽然在形式上看起来很相似,但似然性是用来在已知数据的情况下评估和比较不同模型参数的,而概率是用来预测在已知模型参数的情况下,某个或某些特定事件发生的可能性。

\section{试解释最大似然估计(Maximum Likelihood Estimation)的概念}

最大似然估计(Maximum Likelihood Estimation, MLE)是一种在统计模型中估计模型参数的方法。最大似然估计的核心思想是选择使得观察到的数据出现概率(似然性)最大的参数值。换句话说,它寻找的参数值能最大程度地“解释”观察到的数据。

在数学上,最大似然估计涉及到似然函数,这是一个关于参数的函数,表示在这些参数下观察到当前数据的概率。MLE的过程如下:

\begin{enumerate}
    \item 建立模型:首先确定一个包含未知参数的概率模型。这个模型是用来描述数据生成过程的。
    \item 定义似然函数:接着定义一个似然函数,这个函数是未知参数的函数,表示在不同参数值下得到实际观测数据的概率。
    \item 计算最大似然函数:然后,找到一个参数值(或一组参数值),使得这个似然函数达到最大值。这就是所谓的最大似然估计。
    \item 求解参数:为了找到使似然函数最大化的参数值,我们通常需要对似然函数进行求导,并解出导数等于零的点。这些点即为似然函数的极值点。在许多情况下,这个极值点就是全局最大值。
\end{enumerate}

最大似然估计是一种非常通用的估计方法,它在许多统计模型中都是可行的,尤其是在模型是良定义的概率模型时。MLE估计的一个重要性质是它在很多情况下是一致的(consistency),即随着样本量的增加,估计值会收敛至真实的参数值。此外,如果条件允许,MLE估计还可以是有效的(efficiency),即在所有无偏估计中具有最小的方差。

\section{假设进行 $n$ 次抛硬币伯努利试验,记正面为 $1$、反面为 $0$,则有 $n$ 个伯努利随机变量 $X_1$,$X_2$,…,$X_n$,设正面出现的概率为 $\pi$,设 $n$ 次实验的结果为($x_1$,$x_2$,…,$x_n$),$x_i$ 为 $0$ 或者 $1$。}

\subsection{请写出随机变量 $X_i$ 的概率质量}

随机变量 $X_i$ 的概率质量为:

$$P(X_i=x_i)=\pi^{x_i}(1-\pi)^{1-x_i}$$

\subsection{请写出 $n$ 次实验的对数形式的似然函数 $L(\pi)=\log(f(x_1,x_2,...,x_n;\pi))$ }

$n$ 次实验的对数形式的似然函数为:

\begin{equation}
    \begin{aligned}
        L(\pi)
        &=\log(f(x_1,x_2,...,x_n;\pi))\\
        &=\log(\prod_{i=1}^n\pi^{x_i}(1-\pi)^{1-x_i})\\
        &=\sum_{i=1}^n\log(\pi^{x_i}(1-\pi)^{1-x_i})\\
        &=\sum_{i=1}^n(x_i\log\pi+(1-x_i)\log(1-\pi))\\
        &=\sum_{i=1}^nx_i\log\pi+\sum_{i=1}^n(1-x_i)\log(1-\pi)\\
    \end{aligned}
\end{equation}

\subsection{请推导出对的 $\pi$ 最大似然估计(Maximum Likelihood Estimation)$\hat{\pi}$ }

对似然函数求导,令导数为零,得:

\begin{equation}
    \begin{aligned}
        \pdf{L(\pi)}{\pi}
        &=\pdf{}{\pi}(\sum_{i=1}^nx_i\log\pi+\sum_{i=1}^n(1-x_i)\log(1-\pi))\\
        &=\sum_{i=1}^n\pdf{}{\pi}(x_i\log\pi+(1-x_i)\log(1-\pi))\\
        &=\sum_{i=1}^n(x_i\pdf{}{\pi}\log\pi+(1-x_i)\pdf{}{\pi}\log(1-\pi))\\
        &=\sum_{i=1}^n(x_i\df{1}{\pi}+(1-x_i)\df{-1}{1-\pi})\\
        &=\sum_{i=1}^n(\df{x_i}{\pi}-\df{1-x_i}{1-\pi})\\
        &=\sum_{i=1}^n(\df{x_i-\pi}{\pi(1-\pi)})\\
        &=0\\
    \end{aligned}
\end{equation}

解得:

$$\hat{\pi}=\df{\sum_{i=1}^nx_i}{n}$$

\subsection{试推导出最大似然估计 $\hat{\pi}$ 的方差 Var($\hat{\pi}$)}

\begin{equation}
    \begin{aligned}
        Var(\hat{\pi})
        &=Var(\df{\sum_{i=1}^nx_i}{n})\\
        &=\df{1}{n^2}Var(\sum_{i=1}^nx_i)\\
        &=\df{1}{n^2}\sum_{i=1}^nVar(x_i)\\
        &=\df{1}{n^2}\sum_{i=1}^n\pi(1-\pi)\\
        &=\df{1}{n^2}n\pi(1-\pi)\\
        &=\df{\pi(1-\pi)}{n}\\
    \end{aligned}
\end{equation}

\subsection{抛 5 次硬币,4 次为正面,1 次为背面,求最大似然估计 $\hat{\pi}$ 及其方差 Var($\hat{\pi}$) ;抛 500 次硬币,400 次为正面,100 次为背面,求最大似然估计 $\hat{\pi}$ 及其方差 Var($\hat{\pi}$) }

\subsubsection{抛 5 次硬币,4 次为正面,1 次为背面}

$$\hat{\pi}=\df{\sum_{i=1}^nx_i}{n}=\df{4}{5}=0.8$$

$$Var(\hat{\pi})=\df{\pi(1-\pi)}{n}=\df{0.8\times0.2}{5}=0.032$$

\subsubsection{抛 500 次硬币,400 次为正面,100 次为背面}

$$\hat{\pi}=\df{\sum_{i=1}^nx_i}{n}=\df{400}{500}=0.8$$

$$Var(\hat{\pi})=\df{\pi(1-\pi)}{n}=\df{0.8\times0.2}{500}=0.00032$$

\section{中国与阿根廷乒乓球国家队举行比赛,设中国队取胜的概率 $P$=0.95,那么}

\subsection{中国队取胜的发生比(odds)为多少?对数发生比为多少?}

中国队取胜的发生比(odds)为:

$$\df{P}{1-P}=\df{0.95}{1-0.95}=19$$

对数发生比为:

$$\log(\df{P}{1-P})=\log(\df{0.95}{1-0.95})=\log(19)=2.944$$

\subsection{试解释在广义线性模型中使用对数发生比对的优势}

在广义线性模型中使用对数发生比对的优势是,它可以将发生比的线性组合转化为对数发生比的线性组合,从而使得模型的形式更加简洁。

\section{设 $Y_i$ 为服从泊松分布的独立随机变量,期望为 $\mu_i$,广义线性模型表示为 $$\log(\mu_i)=\beta_0+\beta_1x_i$$ 其中 $x_i$ 和 $y_i$ 为获得的观测样本值,$x_i$ 为自变量,$y_i$ 为因变量,即随机变量 $Y_i$ 的取值,$i=1,2,...,n$,$n$为总样本数 }

\subsection{请分别解释广义线性模型的随机成分、系统性成分、链接函数是什么}

\begin{itemize}
    \item 随机成分:随机成分是指模型中的因变量 $Y_i$,它是一个随机变量,服从某种概率分布。
    \item 系统性成分:系统性成分是指模型中的自变量 $x_i$,它是一个确定的变量。
    \item 链接函数:链接函数是指将随机成分和系统性成分联系起来的函数,它是一个单调可微函数。
\end{itemize}

\subsection{设 $\beta=(\beta_0\ \ \ \ \beta_1)^T$ ,请写出对数形式的似然函数 $L(\beta)=\log(f(y_1,y_2,...,y_n;\beta))$}

\begin{equation}
    \begin{aligned}
        L(\beta)
        &=\log(f(y_1,y_2,...,y_n;\beta))\\
        &=\log(\prod_{i=1}^nf(y_i;\beta))\\
        &=\sum_{i=1}^n\log(f(y_i;\beta))\\
        &=\sum_{i=1}^n\log(\df{e^{-\mu_i}\mu_i^{y_i}}{y_i!})\\
        &=\sum_{i=1}^n(-\mu_i+y_i\log\mu_i-\log(y_i!))\\
        &=\sum_{i=1}^n(-e^{\beta_0+\beta_1x_i}+y_i(\beta_0+\beta_1x_i)-\log(y_i!))\\
    \end{aligned}
\end{equation}

\subsection{请写出对模型参数 $\hat{\beta}=(\hat{\beta_0}\ \ \ \ \hat{\beta_1})^T$ 的最大似然估计方程}

对似然函数求导,令导数为零,得:

\begin{equation}
    \begin{aligned}
        \pdf{L(\beta)}{\beta}
        &=\pdf{}{\beta}(\sum_{i=1}^n(-e^{\beta_0+\beta_1x_i}+y_i(\beta_0+\beta_1x_i)-\log(y_i!)))\\
        &=\sum_{i=1}^n\pdf{}{\beta}(-e^{\beta_0+\beta_1x_i}+y_i(\beta_0+\beta_1x_i)-\log(y_i!))\\
        &=\sum_{i=1}^n(-e^{\beta_0+\beta_1x_i}+y_i(\beta_0+\beta_1x_i)-\log(y_i!))\\
        &=\sum_{i=1}^n(-e^{\beta_0+\beta_1x_i}+y_ix_i)\\
        &=0\\
    \end{aligned}
\end{equation}

解得:

$$\hat{\beta_1}=\df{\sum_{i=1}^ny_ix_i-\sum_{i=1}^ne^{\beta_0+\beta_1x_i}}{\sum_{i=1}^nx_i^2}$$

$$\hat{\beta_0}=\df{\sum_{i=1}^ny_i-\hat{\beta_1}\sum_{i=1}^nx_i}{n}$$

\end{document}