\documentclass{article}
\usepackage[UTF8]{ctex}  % 使用中文支持包
\usepackage[a4paper, margin=1in]{geometry}  % 设置纸张大小和边距
\usepackage{anyfontsize}  % 解决字体大小报错问题
\usepackage{fancyhdr}  % 设置页眉、页脚、页码
\usepackage{longtable}  % 支持长表格

\usepackage{amsmath}  % 数学公式支持
\usepackage{cases}  % 支持联立编号
\usepackage{cite}  % 引用支持

\usepackage{graphicx}  % 插入图片支持
\usepackage{float}  % 设置图片浮动位置
\usepackage{subfigure}  % 插入多图时用子图显示

\usepackage{listings}  % 代码块支持
\usepackage{xcolor}  % 设置代码块颜色

\usepackage[hyphens]{url}  % 支持链接换行
\usepackage{hyperref}  % 超链接支持
\usepackage{lastpage}  % 添加lastpage包

\usepackage{gbt7714}  %国标参考文献

\hypersetup{
    hidelinks,
    colorlinks=true,
    allcolors=black,
    pdfstartview=Fit,
    breaklinks=true
}

\title{聚变能源概论-第三讲作业}
\author{\LaTeX\ by\ Jerry\ }
\date{\today}
\pagenumbering{arabic}

\begin{document}
\pagestyle{fancy}

\fancyhead[L]{Jerry}
\fancyhead[C]{聚变能源概论-第三讲作业}
\fancyhead[R]{\today}
\fancyfoot[C]{Page \thepage/\pageref{LastPage}}

\section*{在下列两组参数下计算聚变反应功率、韧致辐射功率、回旋辐射功率:}

(1) $\text{D}-\text{T}$: $n = 10^{20} \text{m}^{-3}$, $B = 6 \text{T}, T = 10 \text{keV}$;

(2) $\text{D}-^3\text{He}$: $n = 10^{20} \text{m}^{-3}$, $B = 6 \text{T}, T = 100 \text{keV}$. \emph{并说明回旋辐射功率的再吸收效应是如何影响聚变功率平衡实现的.}

聚变反应的功率计算公式为

$$P_f = n_1 n_2 \langle \sigma v \rangle \cdot E_f$$

轫致辐射计算公式为

$$S_B = 1.625 \times 10^{-38} Z_{eff} n_e^2\sqrt{T_e}, Z_{eff} = \frac{\sum_j Z_j^2n_j}{n_e}$$

回旋辐射功率计算公式

$$S_c = 6.21 \times 10^{-20} B^2n_eT_e$$

(1) $\text{D}-\text{T}$: $n = 10^{20} \text{m}^{-3}$, $B = 6 \text{T}, T = 10 \text{keV}$, 此时反应率系数取$\langle \sigma v \rangle = 1.13 \times 10^{-22} \text{m}^3/\text{s}$

假设D, T粒子数密度相等, 则聚变反应功率

$$P_f = \frac{n^2}{4} \langle \sigma v \rangle \cdot E_f = 0.744 \text{MW}/\text{m}^3$$

对于D, T等离子体有效电荷数为: $$Z_{eff} = 1$$

轫致辐射功率为:

$$S_B = 1.625 \times 10^{-38} \times 1 \times 10^{40} \times \sqrt{10^4}\text{W}/\text{m}^3 = 16.25 \text{kW}/\text{m}^3$$

回旋辐射功率为:

$$S_C = 6.21 \times 10^{-20} \times 36 \times 10^{20} \times 10^4\text{W}/\text{m}^3 = 2.236 \text{kW}/\text{m}^3$$

(2) $\text{D}-^3\text{He}$: $n = 10^{20} \text{m}^{-3}$, $B = 6 \text{T}, T = 100 \text{keV}$.

假设D, He3粒子数密度相等, 则粒子数密度为

$$n_D = n_{He} = \frac{n}{3}$$

D, He3聚变反应功率为:

$$P_f = \frac{n^2}{9} \langle \sigma v \rangle \cdot E_f = 0.56 \text{MW}/\text{m}^3$$

对于D, He3等离子体有效电荷数为: $$Z_{eff} = \frac{4}{3} + \frac{1}{3} = \frac{5}{3}$$

轫致辐射功率为:

$$S_B = 1.625 \times 10^{-38} \times \frac{5}{3} \times 10^{40} \times \sqrt{10^5}\text{W}/\text{m}^3 = 85.65 \text{kW}/\text{m}^3$$

回旋辐射功率为:

$$S_C = 6.21 \times 10^{-20} \times 36 \times 10^{20} \times 10^5\text{W}/\text{m}^3 = 22.36 \text{MW}/\text{m}^3$$

从计算可以看出一般参数下的回旋辐射功率远大于聚变反应功率和轫致辐射功率, 且随温度增长很快, 如果回旋辐射没有很高的再吸收率, 等离子体不可能达到功率平衡.

\end{document}
