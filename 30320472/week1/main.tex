\documentclass{article}
\usepackage[UTF8]{ctex}  % 使用中文支持包
\usepackage[a4paper, margin=1in]{geometry}  % 设置纸张大小和边距
\usepackage{anyfontsize}  % 解决字体大小报错问题
\usepackage{fancyhdr}  % 设置页眉、页脚、页码
\usepackage{longtable}  % 支持长表格

\usepackage{amsmath}  % 数学公式支持
\usepackage{cases}  % 支持联立编号
\usepackage{cite}  % 引用支持

\usepackage{graphicx}  % 插入图片支持
\usepackage{float}  % 设置图片浮动位置
\usepackage{subfigure}  % 插入多图时用子图显示

\usepackage{listings}  % 代码块支持
\usepackage{xcolor}  % 设置代码块颜色

\usepackage[hyphens]{url}  % 支持链接换行
\usepackage{hyperref}  % 超链接支持
\usepackage{lastpage}  % 添加lastpage包

\hypersetup{
    hidelinks,
    colorlinks=true,
    allcolors=black,
    pdfstartview=Fit,
    breaklinks=true
}

\title{聚变能源概论-第一讲作业}
\author{\LaTeX\ by\ Jerry\ }
\date{\today}
\pagenumbering{arabic}

\begin{document}
\pagestyle{fancy}

\fancyhead[L]{Jerry}
\fancyhead[C]{聚变能源概论-第一讲作业}
\fancyhead[R]{\today}
\fancyfoot[C]{Page \thepage/\pageref{LastPage}}

\section*{第一讲作业}

\subsection*{1. 在地球表面, 太阳直射时每平方米辐射功率约为1400W}

\subsubsection*{a. 请推算太阳辐射的总功率(太阳的光度); 如果这些辐射能量均来自质子-质子循环(不考虑 CNO 循环), 那么太阳现有的质量(假设全为质子)还可以维持多少年?}

我们已知常数: 日地距离 $L=149597870\text{km} = 149597870000\text{m}$, 太阳的辐射照度为 $I=1400\text{W}/\text{m}^2$.

合理假设太阳发出的光各向同性, 我们假设有一个以这个距离为半径的球且太阳为球心, 则此球的各个表面均为太阳直射, 且各点功率相同. 

故总功率(太阳的光度)为
$$P = I \cdot S = I \cdot 4 \pi L^2 = 3.9372112713911865 \times 10^{26} \text{W}$$

太阳质量为 $M=1.989 \times 10^{30} \text{kg}$, 质子质量为 $m_p=1.6726219237\times 10^{-27} \text{kg}$. 仅考虑质子-质子循环, 平均每4个质子可以释放 $E_0 = 26.73 MeV$ 的能量. 故太阳释放的能量为
$$E = \frac{E_0}{4} \frac{M}{m_p} = 1.2731698905021498 \times {10}^{45} \text{J}$$

故维持年数为
$$T = \frac{E}{P} = 1024720826720.0635 \approx 1.025 \times {10}^{11} \text{a}$$

\subsubsection*{b. 考虑太阳每天的照射时间、角度、天气因素及转换效率, 估计一个合理的太阳能的使用效率; 按照目前每人每天消耗 52kWh 能量计算, 估算每人需要多大面积的太阳能提供才能满足要求. }

我们已知:
\begin{itemize}
    \item 太阳直射的功率为 $I = 1400\text{W}/\text{m}^2 = 1.4 \text{kW} / \text{m}^2$
    \item 人均每天消耗能量为 $E_0 = 52\text{kWh}$
\end{itemize}

我们作出以下假设:
\begin{itemize}
    \item 当地地区纬度为 $\varphi$, 太阳直射的纬度为 $\phi$, 太阳高度角为 $\theta$, 时角为 $\omega(t) = 15^\circ \times t$
    \item 太阳能板转换效率为 $\eta=0.2$, 天气因素导致的有效照射功率为 $I_{\text{eff}} = 0.8 \times I$
    \item 太阳能板平行于地面放置
\end{itemize}

可以计算出太阳每天的照射时间为 $$D = \frac{24}{\pi} \arccos(\tan(\varphi) \tan(\phi))$$.

在任意时刻, 太阳高度角 $\theta$ 为 $$\sin(\theta) = \sin(\varphi) \sin(\phi) + \cos(\varphi) \cos(\phi) \cos(\omega(t))$$.

在任意时刻, 太阳能板接收到的功率与太阳高度角的正弦成正比: $$P = I_{\text{eff}} \sin(\theta) = I_{\text{eff}} (\sin(\varphi) \sin(\phi) + \cos(\varphi) \cos(\phi) \cos(\omega(t)))$$.

故单位面积太阳能板每天接收到的总能量为

$$E_{\text{total}} = \int_{-\frac{D}{2}}^{\frac{D}{2}} P \, \text{d}t = I_{\text{eff}} \int_{-\frac{D}{2}}^{\frac{D}{2}} (\sin(\varphi) \sin(\phi) + \cos(\varphi) \cos(\phi) \cos(\omega(t))) \, \text{d}t$$

故每人需要的太阳能板面积为

$$S = \frac{E_0}{E_{\text{total}} \cdot \eta} = \frac{E_0}{0.8 \cdot \eta \cdot I \cdot \int_{-\frac{D}{2}}^{\frac{D}{2}} (\sin(\varphi) \sin(\phi) + \cos(\varphi) \cos(\phi) \cos(\omega(t))) \, \text{d}t}$$

为了方便计算, 我们假设太阳直射赤道, 即 $\phi = 0^\circ$, 且我们位于北纬 $45^\circ$ 的地区, 即 $\varphi = 45^\circ$, 则太阳每天的照射时间为 $D = 12\text{h}$.

代入数据, 我们可以计算单位面积太阳能板每天接收到的总能量为 $$E_{\text{total}} = 9.5023 \text{kWh}$$.

从而计算出每人需要的太阳能板面积为 $$S = \frac{E_0}{E_{\text{total}} \cdot \eta} = 27.36 \text{m}^2$$

\subsection*{2. 请大家畅想无限、清洁、安全、廉价的聚变能实现后, 还有哪些想象空间?}

实现聚变能后,我们将拥有几乎无限的清洁能源,这将彻底革新全球能源结构,推动经济和技术的巨大变革。由于能源成本的显著降低,各行业的生产力将得到大幅提升,新兴产业和技术迅速崛起,推动工业的全面升级。电动汽车、飞机和船舶将更加普及,甚至可能出现核聚变驱动的交通工具,减少对化石燃料的依赖,为全球交通革命注入新的活力。同时,廉价而充足的能量将为深空探索提供强有力的支持,加速人类对其他星球的探测与定居。

此外,环境保护将有显著提升。温室气体排放大幅减少,气候变化的压力得到缓解,空气质量和生态系统显著改善。发展中国家则有机会获得廉价和可靠的能源供应,缩小全球经济差距,提升生活质量和社会福利。这将促进全球经济的平衡发展,推动人类社会向更加公平和可持续的方向迈进,为全球社会带来深远而积极的影响。

\end{document}
