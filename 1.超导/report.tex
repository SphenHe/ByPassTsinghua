% 大学物理实验报告

\documentclass[UTF8]{ctexart}

\usepackage{amsmath}        %数学公式
\usepackage{cases}          %联立编号
\usepackage{cite}           %引用

\usepackage{graphicx}       %插入图片
\usepackage{float}          %设置图片浮动位置
\usepackage{subfigure}      %插入多图时用子图显示

\usepackage{anyfontsize}    %解决一个奇怪的字体大小报错问题
\usepackage{fancyhdr}       %页眉、页脚、页码
\usepackage[a4paper, margin=1in]{geometry}    %纸张大小


\setlength{\headheight}{16pt}
\pagestyle{fancy}
\fancyhf{}


\title{高温超导材料转变温度测定实验报告}
\author{\LaTeX\ by\ Jerry\ }
\date{\today}
\pagenumbering{arabic}

\begin{document}

\fancyhead[L]{Jerry}
\fancyhead[C]{高温超导材料转变温度测定实验}
\fancyfoot[C]{\thepage}

\maketitle
\tableofcontents
\newpage

\section{原始数据}

    \subsection{超导样品电阻测量}

    1. 数字万用表两条外引线电阻测量:$R_{outwire}=0.10\Omega$

    2. 超导盒内样品与内在引线、外表引线串联电阻之和:$R_{sum}=0.50\Omega$

    3. 超导盒内样品与内在引线电阻值:$R_{wire}+R_{SC}=0.40\Omega$

    4. CH3工作模式:恒压或恒流:恒流,输出电压$V=643$mV,输出电流$I=1000$mA,总负载电阻$R=0.643$m$\Omega$,$R_{SC}=0.413$m$\Omega$

    \subsection{用电流换向法消除乱真电势的影响}

    $V_{Meas1}$(3档)$=0.414$mV,$V_{Meas2}$(4档)$=-0.416$mV,$V$乱真电势$=1\mu$V

    (与超导样品上的电压相比,乱真电势很小,而且会随温度变化而改变,由于实验过程中换向开关不易频繁操作,同学们测出室温下的乱真电势和超导态下的乱真电势即可)

    \subsection{用铂电阻温度计测量温度}

    1. 限流电阻$R_0=10$k$\Omega$

    2. 计算四位半数字电压表的显示初值$V_{calc}=109.70$mV,(室温:24.9$^{\circ}$C,$R_{Pt}=109.70\Omega$)

    3. 实际四位半电压表数值$V_{real}=108.78$mV,

    4. CH1工作模式:恒压或恒流:恒压,输出电压$V=10.003$V,输出电流$I=1$mA

    \subsection{用电磁感应法测超导样品对感应电压的影响}

    正弦信号$f=700$Hz,直流偏置offset=0,$V_{pp}=2.00$V,$V_{induction}=10.62$mV

    \subsection{降温实验}

    \subsubsection{降温}

    1. 降温数据:(注意:样品电压突降至0微伏的极小值并稳定不变后,继续每隔0.5分钟测一次,共5组左右)

    \begin{table}[H]
        \centering
        \begin{tabular}{|l|l|l|l|l|l|l|l|l|l|l|}
        \hline
            温度($^{\circ}$C) & 20 & 10 & 0 & -10 & -20 & -30 & -40 & -50 & -60 & -70 \\ \hline
            $V_{Pt}$(mV) & 107.75 & 103.84 & 99.95 & 95.97 & 92.10 & 88.20 & 84.24 & 80.27 & 76.35 & 72.38 \\ \hline
            $V_{Sample}$($\mu$V) & 413 & 409 & 404 & 397 & 387 & 378 & 369 & 359 & 327 & 311 \\ \hline
            $V_B$(mV) & 10.63 & 10.60 & 10.60 & 10.64 & 10.61 & 10.50 & 10.44 & 10.39 & 10.24 & 9.67 \\ \hline
            \hline
            温度($^{\circ}$C) & -80 & -90 & -100 & -110 & -120 & -130 & -140 & -150 & -160 & -170 \\ \hline
            $V_{Pt}$(mV) & 68.38 & 64.36 & 60.34 & 56.28 & 52.25 & 48.22 & 44.15 & 40.07 & 35.96 & 31.80 \\ \hline
            $V_{Sample}$($\mu$V) & 296 & 281 & 264 & 250 & 233 & 215 & 197 & 178 & 158 & 140 \\ \hline
            $V_B$(mV) & 11.01 & 11.35 & 11.44 & 11.43 & 11.60 & 11.72 & 11.80 & 11.89 & 11.99 & 12.13 \\ \hline
            \hline
            温度($^{\circ}$C) & -175 & -176 & -177 & -178 & -179 & -180 & -181 & -185 & -188 & -190 \\ \hline
            $V_{Pt}$(mV) & 29.74 & 29.46 & 29.06 & 28.62 & 28.19 & 27.79 & 27.41 & 25.60 & 24.38 & 23.64 \\ \hline
            $V_{Sample}$($\mu$V) & 129 & 94 & 21 & 4 & 2 & 2 & 1 & 1 & 2 & 2 \\ \hline
            $V_B$(mV) & 12.16 & 12.18 & 12.06 & 11.41 & 11.10 & 10.97 & 10.87 & 10.69 & 10.66 & 10.66 \\ \hline
        \end{tabular}
    \end{table}

    超导态下:$V_{Meas1}$(3档)=0.001mV,$V_{Meas2}$(4档)=0.002mV,$V$乱真电势=1.5$\mu$V,简化起见,乱真电势修正统一以此值为准 。

    \subsubsection{升温}

    \begin{table}[H]
        \centering
        \begin{tabular}{|l|l|l|l|l|l|l|l|l|l|l|}
        \hline
            温度($^{\circ}$C) & -190 & -185 & -180 & -175 & -170 & -165 & -163 & -162 & -161 & -160 \\ \hline
            $V_{Pt}$(mV) & 23.65 & 25.71 & 27.76 & 29.84 & 31.85 & 33.94 & 34.76 & 35.15 & 35.61 & 36.00 \\ \hline
            $V_{Sample}$($\mu$V) & 2 & 2 & 3 & 3 & 2 & 3 & 2 & 3 & 3 & 5 \\ \hline
            $V_B$(mV) & 10.70 & 10.69 & 10.70 & 10.73 & 10.76 & 10.91 & 11.06 & 11.16 & 11.37 & 11.62 \\ \hline
            \hline
            温度($^{\circ}$C) & -159 & -158 & -157 & -156 & -155 & -154 & -153 & -152 & -151 & -150 \\ \hline
            $V_{Pt}$(mV) & 36.38 & 36.79 & 37.20 & 37.64 & 38.04 & 38.44 & 38.84 & 39.27 & 39.65 & 40.08 \\ \hline
            $V_{Sample}$($\mu$V) & 15 & 57 & 121 & 131 & 132 & 134 & 135 & 137 & 139 & 140 \\ \hline
            $V_B$(mV) & 12.04 & 12.21 & 12.21 & 12.20 & 12.19 & 12.19 & 12.17 & 12.17 & 12.16 & 12.16 \\ \hline
            \hline
            温度($^{\circ}$C) & -149 & -148 & -147 & -146 & -145 & -144 & -143 & -142 & -141 & -140 \\ \hline
            $V_{Pt}$(mV) & 40.49 & 40.90 & 41.32 & 41.72 & 42.12 & 42.54 & 42.93 & 43.34 & 43.72 & 44.13 \\ \hline
            $V_{Sample}$($\mu$V) & 141 & 142 & 143 & 146 & 147 & 148 & 150 & 151 & 153 & 154 \\ \hline
            $V_B$(mV) & 12.15 & 12.14 & 12.13 & 12.12 & 12.12 & 12.11 & 12.10 & 12.09 & 12.09 & 12.08 \\ \hline
        \end{tabular}
    \end{table}

\section{数据处理}

    \subsection{计算样品温度}

    用测得的铂电阻电压数据,计算样品温度$t$

    计算样品电阻$R_t$和乱真电势。比较乱真电势与超导转变后的样品电压的数量级大小。

    \subsection{做图}

    分别画出样品在降温、升温过程中的电阻—温度特性曲线和感应电压—温度曲线。

    从升温时的电阻—温度曲线和感应电压—温度曲线中分别求出转变温度$T^{\circ}$C和转变宽度$\Delta T^{\circ}$C

    \subsection{分析讨论}

    对实验现象、测量结果等作一定的分析讨论。

    \subsubsection{实验现象}

    \subsubsection{测量结果}

\section{思考题}

    设想一个中空的超导圆管样品,转变温度为$T^{\circ}$C。在温度高于$T^{\circ}$C时,对圆管施加外磁
    场。而后将温度降低到转变温度$T^{\circ}$C以下。在实验报告中分析样品的磁感应强度和感应电流情况。
    撤掉外磁场后,样品的磁感应强度和感应电流如何变化:

\newpage
\section{原始数据}
    \begin{center}
        \begin{figure}[H] %H为当前位置,!htb为忽略美学标准,htbp为浮动图形
            \centering %图片居中
            \includegraphics[width=0.8\textwidth]{img/OriginalData1.jpg} %插入图片,[]中设置图片大小,{}中是图片文件名
            \caption{原始数据1} %最终文档中希望显示的图片标题
            \label{原始数据-1} %用于文内引用的标签
        \end{figure}
        \begin{figure}[H] %H为当前位置,!htb为忽略美学标准,htbp为浮动图形
            \centering %图片居中
            \includegraphics[width=0.8\textwidth]{img/OriginalData2.jpg} %插入图片,[]中设置图片大小,{}中是图片文件名
            \caption{原始数据2} %最终文档中希望显示的图片标题
            \label{原始数据-2} %用于文内引用的标签
        \end{figure}
    \end{center}
\end{document}